\documentclass{beamer}

\usepackage{outlines}

\usetheme{Singapore}

\AtBeginSection[]
{
    \begin{frame}
        \frametitle{Table of Contents}
        \tableofcontents[currentsection]
    \end{frame}
}

\title{Implementing TULI5 in Elixir}
\author{Astrid Yu}
\institute{CSC 430}
\date{2022-03-11}

\begin{document}

\frame{\titlepage}

\begin{frame}{Elixir in a nutshell}
    \begin{columns}[T]
    \begin{column}{.48\textwidth}
        \begin{outline}
            \1 Functional, dynamically-typed, general-purpose language. A lot like untyped Racket.
            \1 Released in 2012
            \1 Comes with a REPL (\texttt{iex})
            \1 Lots of interop with Erlang
                \2 It's made in Erlang
                \2 Compiles to Erlang VM bytecode
        \end{outline}
    \end{column}
    \begin{column}{.48\textwidth}
        \centering
        \includegraphics[width=0.8\linewidth]{./elixir.png}
    \end{column}
    \end{columns}
\end{frame}

\begin{frame}{StackOverflow Survey 2021}
    \begin{outline}
        \1 Elixir pays the \textbf{3rd} most! (\$80,077) Tied with Erlang, likely because of how connected they are
        \1 However, it is the \textbf{30th} most used language
        \1 It is the \textbf{4th} most loved language (72\% love vs 28\% dread)
    \end{outline}
    These statistics seem similar to lots of other functional programming languages
\end{frame}

\begin{frame}{Language features}
    \begin{outline}
        \1 Standard stuff: numbers, strings, variables, closures, tuples...
        \1 Everything is an expression
        \1 Can be compiled (.ex) OR interpreted (.exs)
        \1 Maps (basically dictionaries)
    \end{outline}
\end{frame}

\begin{frame}{Probably stolen from Lisp}
    \begin{outline}
        \1 ``Atoms,'' which are just symbols by another name
            \2 Elixir's \texttt{:abc} is Racket's \texttt{'abc}
        \1 Linked lists
        \1 AST-based macros, along with a certain \texttt{quote/1} and \texttt{unquote/1}
    \end{outline}
\end{frame}

\begin{frame}{Constructing TULI in Elixir}
    \texttt{[:fn, [:x], [:+, [:-, :x, 9], 10]]}
\end{frame}

\begin{frame}{What I've made}
    \begin{outline}
        \1 A working interpreter and parser
        \1 I didn't bother to make most of the builtin environment
        \1 Unknown how well it deals with edge cases, but oh well
    \end{outline}
\end{frame}

\begin{frame}
    \Huge
    \centering
    Demo
\end{frame}

\begin{frame}{Would I take a job writing Elixir?}
    \begin{outline}
        \1 Maybe, it seems kinda interesting
        \1 It does seem to have nice tooling
        \1 But also, I don't really like dynamically typed languages
    \end{outline}
\end{frame}

\begin{frame}
    \frametitle{Thanks for listening!}
    
    {\Huge \textbf{Any questions?}}

    \vspace{1cm}

    {\large \textbf{Astrid Yu}}

    Website: \href{https://astrid.tech}{astrid.tech}

    Email: \href{mailto:astrid@astrid.tech}{astrid@astrid.tech}
    
    Github: \href{https://github.com/astralbijection}{@astralbijection}
\end{frame}

\end{document}

\documentclass{beamer}

\usepackage{verbatim}

\usetheme{AnnArbor}

\AtBeginSection[]
{
    \begin{frame}
        \frametitle{Table of Contents}
        \tableofcontents[currentsection]
    \end{frame}
}

\title{archlinux-2022.01.01-x86\_64.iso Let's Play}
\author{Astrid Yu}
\institute{Cal Poly Linux Users Group}
\date{2022-01-09}

\begin{document}

\frame{\titlepage}

\begin{frame}{Why Arch?}
    \begin{itemize}
        \item Rolling release means you have very bleeding-edge packages, all you gotta do is just \texttt{pacman -Sy <package>}
        \item Even if pacman doesn't have it, AUR probably does!
        \item You get to learn a lot about the various tools that modern Linux distros run on
        \item Manual installation process may be hard, but you can do almost anything you can imagine!
            \begin{itemize}
                \item \href{https://gist.github.com/motorailgun/cc2c573f253d0893f429a165b5f851ee}{Some person got Arch and Windows installed on the same NTFS partition using Arch to make an extremely cursed computer}
            \end{itemize} \pause
        \item You can finally say the magic words: ``I use arch btw''
    \end{itemize}
\end{frame}

\begin{frame}{How to install Arch}
    \Huge
    \href{https://wiki.archlinux.org/title/Archinstall}{Use archinstall, the built-in installer}
\end{frame}

\newcommand{\thankyouframe}{
    \frametitle{Thanks for listening!}
    
    {\Huge \textbf{Any questions?}}

    {\small The source code for this presentation is located at \url{https://github.com/astralbijection/presentations/tree/main/cplug-install-arch}}

    \vspace{1cm}

    {\large \textbf{Astrid Yu}}

    Website: \href{https://astrid.tech}{astrid.tech}

    Email: \href{mailto:astrid@astrid.tech}{astrid@astrid.tech}
    
    Github: \href{https://github.com/astralbijection}{@astralbijection}
}

\begin{frame}
    \thankyouframe \pause

    \centering
    \huge
    \textbf{I'm just kidding}
\end{frame}

\begin{frame}{How to install Arch \textit{without a guided installer}}
    Okay, so maybe archinstall makes things too easy.
    \begin{itemize}
        \item How do we install it the classical and very manual way?
        \item How do we get that sense of pride and accomplishment that comes with manually getting it to work?
    \end{itemize}
\end{frame}

\begin{frame}{Outline of talk}
    \tableofcontents
\end{frame}

\begin{frame}{Following along from home}
    \begin{itemize}
        \item I may \textit{look} like a gigachad supergenius doing this quickly, but I've done it before
        \item Follow this guide from the Arch Linux wiki: \url{https://wiki.archlinux.org/title/installation_guide}
        \item I'm doing this in a VM, which is kind of an idealized environment. Your hardware may be harder to set up!
    \end{itemize}
\end{frame}

\section{Making your Live USB}

\begin{frame}{Download Arch}
    \large

    \url{https://archlinux.org/download/}
\end{frame}

\begin{frame}{Flash Arch onto a USB}

    \begin{block}{Mac and Linux users}
        \begin{enumerate}
            \item Use lsblk to list disks and find the USB you want to flash onto
            \item \texttt{sudo dd bs=16M if=<your arch iso> of=<the USB you want> status=progress}
        \end{enumerate}

        \textbf{WARNING!} If you type in the wrong thing for \textit{of=}, you \textit{will destroy your disk.} It's called dd (disk destroyer) for a reason!\footnote{It's not actually called disk destroyer, but the memes all say it is.}
    \end{block}

    \begin{block}{Windows users}
        Rufus (found at \url{https://rufus.ie/en/}) is one of the better disk imagers I know of.
\end{block}
\end{frame}

\begin{frame}
    \frametitle{Boot into Arch on your target computer}

    \begin{enumerate}
        \item Stick the USB into your computer
        \item Boot into the USB (may need BIOS)
    \end{enumerate}
\end{frame}

\begin{frame}
    \Huge
    \centering

    Thankfully, I've gone through all these steps to this VM!
\end{frame}

\section{Setting up your disks}

\begin{frame}{Why?}
    
    \begin{itemize}
        \item Files are a big fat lie
        \item On disk, everything is arrangements of bits
    \end{itemize}
\end{frame}

\begin{frame}{Planning our disk}

    We want the following partitions:
    \begin{itemize}
        \item an \textbf{EFI} partition to install our bootloader in (FAT, 550MB)
        \item a \textbf{/ (root)} partition to install our system in (ext4, 20GB)
        \item a \textbf{/home} partition to store user files in (ext4, 30GB)
    \end{itemize}
\end{frame}

\begin{frame}{Making partition tables}
    Use \texttt{fdisk}
\end{frame}

\begin{frame}{Formatting our filesystems}
    Use \texttt{mkfs.ext4} and \texttt{mkfs.vfat}

    \texttt{e2label /dev/sda2 arch\_os}
\end{frame}

\begin{frame}{Mounting our filesystems}
    Use the \texttt{mount} command to mount root:

    \texttt{mount /dev/sda2 /mnt}

    Then, mount additional partitions
    \begin{enumerate}
        \item \texttt{mkdir -p /mnt/boot /mnt/home}
        \item \texttt{mount /dev/sda1 /mnt/boot}
        \item \texttt{mount /dev/sda3 /mnt/home}
    \end{enumerate}
\end{frame}

\section{Installing the system and chrooting}

\begin{frame}{Why?}
    \begin{itemize}
        \item You literally need to do this in order to have an Arch system
        \item \texttt{chroot} means ``change root''; you do this to treat a different filesystem as your root so \texttt{pacman -S} goes onto the system you're installing, even if you're technically in the live USB
    \end{itemize}
\end{frame}

\begin{frame}{Run pacstrap}
    \texttt{pacstrap /mnt base linux linux-firmware e2fsprogs dosfstools bash neovim iproute2 networkmanager systemd efibootmgr}

    \begin{alertblock}{Note}
        Not all of these things are mandatory. \texttt{neovim} and everything after it can potentially be swapped out for others.
    \end{alertblock}
\end{frame}

\begin{frame}{Make your fstab}
    \texttt{genfstab -U /mnt >> /mnt/etc/fstab}
\end{frame}

\begin{frame}{\texttt{chroot} into your mounted disk}
    \texttt{arch-chroot /mnt}
\end{frame}

\begin{frame}{Random system stuff}
    \begin{itemize}
        \item \texttt{ln -sf /usr/share/zoneinfo/US/Pacific /etc/localtime}
        \item \texttt{hwclock --systohc}
        \item \texttt{echo archlinux-lets-play > /etc/hostname}
        \item \texttt{passwd}
    \end{itemize}
\end{frame}

\begin{frame}{Locales}
    \begin{itemize}
        \item \texttt{locale-gen}
        \item \texttt{echo LANG=en\_US.UTF-8 > /etc/locale.conf}
        \item \texttt{echo KEYMAP=us > /etc/vconsole.conf}
    \end{itemize}
\end{frame}

\section{Installing your bootloader}

\begin{frame}{Why?}
    Without a bootloader, you can't boot into Arch, defeating the whole purpose of all of this
\end{frame}

\begin{frame}[fragile]{Install systemd-boot}
    Run \texttt{bootctl install}

    \begin{exampleblock}{\texttt{/boot/loader/loader.conf}}
        \begin{verbatim}
default  arch.conf
timeout  4
console-mode max
editor   no\end{verbatim}
    \end{exampleblock}

    \begin{exampleblock}{\texttt{/boot/loader/entries/arch.conf}}
        \begin{verbatim}
title   Arch Linux
linux   /vmlinuz-linux
initrd  /initramfs-linux.img
options root="LABEL=arch_os" rw\end{verbatim}
    \end{exampleblock}
\end{frame}

\begin{frame}{Final Notes}
\begin{itemize}
    \item Remember that this is on a VM! It might be harder on your hardware! \pause
    \item Don't just follow this presentation, read the Arch wiki too! \url{https://wiki.archlinux.org/title/installation_guide} \pause
    \item Wanna do more than just have a TTY? Read the Arch wiki, it will show you how to install KDE and other stuff! \pause
\end{itemize}
\end{frame}

\begin{frame}
    \thankyouframe
\end{frame}

\end{document}

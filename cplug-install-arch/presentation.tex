\documentclass{beamer}

\title{archlinux-2022.01.01-x86\_64.iso Let's Play}
\author{Astrid Yu}
\institute{Cal Poly Linux Users Group}
\date{2022-??-??}

\begin{document}

\frame{\titlepage}

\begin{frame}
    \frametitle{Why Arch?}
    \begin{itemize}
        \item Rolling release means you have very bleeding-edge packages, all you gotta do is just \texttt{pacman -Sy <package>}
        \item Even if pacman doesn't have it, AUR probably does!
        \item You get to learn a lot about the various tools that modern Linux distros run on
        \item Manual installation process may be hard, but you can do almost anything you can imagine!
            \begin{itemize}
                \item \href{https://gist.github.com/motorailgun/cc2c573f253d0893f429a165b5f851ee}{Some person got Arch and Windows installed on the same NTFS partition using Arch to make an extremely cursed computer}
            \end{itemize} \pause
        \item You can finally say the magic words: ``I use arch btw''
    \end{itemize}
\end{frame}

\begin{frame}
    \frametitle{How to install Arch}
    \Huge
    \href{https://wiki.archlinux.org/title/Archinstall}{Use archinstall, the built-in installer}
\end{frame}

\newcommand{\thankyouframe}{
    \frametitle{Thanks for listening!}

    {\large \textbf{Astrid Yu}}

    Website: \href{https://astrid.tech}{astrid.tech}

    Email: \href{mailto:astrid@astrid.tech}{astrid@astrid.tech}
    
    Github: \href{https://github.com/astralbijection}{@astralbijection}
}

\begin{frame}
    \thankyouframe \pause

    \centering
    \huge
    \textbf{I'm just kidding}
\end{frame}

\begin{frame}
    \frametitle{How to \textit{actually} install Arch}
    Okay, so maybe archinstall makes things too easy.
    \begin{itemize}
        \item How do we install it the classical and very manual way?
        \item How do we get that sense of pride and accomplishment that comes with manually getting it to work?
    \end{itemize}
\end{frame}

\begin{frame}
    \frametitle{Download Arch}

    \large

    \url{https://archlinux.org/download/}
\end{frame}

\begin{frame}
    \frametitle{Flash Arch onto a USB}

    \begin{block}{Mac and Linux users}
        \begin{enumerate}
            \item Use lsblk to list disks and find the USB you want to flash onto
            \item \texttt{sudo dd bs=16M if=<your arch iso> of=<the USB you want> status=progress}
        \end{enumerate}

        \textbf{WARNING!} If you type in the wrong thing for \textit{of=}, you \textit{will destroy your disk.} It's called dd (disk destroyer) for a reason!\footnote{It's not actually called disk destroyer, but the memes all say it is.}
    \end{block}

    \begin{block}{Windows users}
        Rufus (found at \url{https://rufus.ie/en/}) is one of the better disk imagers I know of.
\end{block}
\end{frame}

\begin{frame}
    \frametitle{Boot into Arch on your target computer}

    \begin{enumerate}
        \item Stick the USB into your computer
        \item Boot into the USB (may need BIOS)
    \end{enumerate}
\end{frame}

\begin{frame}
    \Huge
    \centering

    Thankfully, I've gone through all these steps to this VM!
\end{frame}

\begin{frame}
    \frametitle{Disk formatting}
\end{frame}

\end{document}

\documentclass{beamer}

\usepackage{minted}

\title{NixOS and why you should use it}
\author{Astrid Yu}
\institute{Cal Poly Linux Users Group}
\date{2021}

\begin{document}

\frame{\titlepage}

\begin{frame}
    \frametitle{Traditional way: SSH and run commands}

    \begin{block}{How to do it}
        \begin{enumerate}
            \item SSH into your machine
            \item \texttt{apt install your-service}
            \item \texttt{vi /etc/your-service/config.conf}
            \item \texttt{systemctl enable your-service.service}
            \item Repeat for each service, repeat for each machine
        \end{enumerate}
    \end{block}
\end{frame}

\begin{frame}
    \frametitle{Traditional way: SSH and run commands}

    \begin{alertblock}{Pros}
        \begin{itemize}
            \item Pretty simple
            \item Intuitive, uses built-in tools
        \end{itemize}
    \end{alertblock}

    \pause

    \begin{alertblock}{Cons}
        \begin{itemize}
            \item If you need to migrate you lose all your configs
            \item You'll eventually lose track of what you've modified
        \end{itemize}
    \end{alertblock}
\end{frame}

\begin{frame}
    \frametitle{Shell scripts in a Git repo}
    
    \begin{block}{OwO what's this?}
        \begin{itemize}
            \item Think SSH and Bash on steroids
            \item They have features that help make execution \textbf{idempotent}
            \item If you run something twice, it only gets applied once
        \end{itemize}
    \end{block}
\end{frame}

\begin{frame}[fragile]
    \frametitle{Example: Ansible}

    Create a config
    \begin{minted}{cfg}
# ansible.cfg
[defaults]
inventory = inventory.yml
    \end{minted}
\end{frame}

\begin{frame}[fragile]
    \frametitle{Example: Ansible}

    Declare your inventory of machines
    \begin{minted}{cfg}
# inventory.yml
all:
  myhosts:
    cracktop.p.astrid.tech:
      ansible_host: 192.168.8.41
      ansible_user: root
    thonkpad.p.astrid.tech:
      ansible_host: 192.168.8.42
      ansible_user: root
    badtop.p.astrid.tech:
      ansible_host: 192.168.8.22
      ansible_user: astrid
    ...
    \end{minted}
\end{frame}

\begin{frame}[fragile]
    \frametitle{Example: Ansible}

    Create a playbook:
    \begin{minted}{yaml}
# playbook.yaml
- hosts: myhosts # run on every one of myhosts
  tasks:
    - apt:  # apt
        item: 
          - your-service-1
          - your-service-2
          - ...
        state: present
    - copy:
        src: /your/local/copy
        dest: /etc/your-service/config.conf
    - systemd:
        name: your-service-1.service
        state: started
        enabled: yes
        \end{minted}
\end{frame}

\begin{frame}
    \begin{alertblock}{Pros}
        \begin{itemize}
            \item Pretty simple
            \item Intuitive, uses built-in tools
        \end{itemize}
    \end{alertblock}
    \pause

    \begin{alertblock}{Cons}
        \begin{itemize}
            \item If you need to migrate you lose all your configs
            \item You'll eventually lose track of what you've modified
        \end{itemize}
    \end{alertblock}
\end{frame}

\end{document}
